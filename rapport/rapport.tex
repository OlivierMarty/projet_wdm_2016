\documentclass[a4paper, 12pt]{article}
\usepackage[utf8]{inputenc}
\usepackage[T1]{fontenc}
\usepackage[francais, english]{babel}
\usepackage[a4paper]{geometry}
\geometry{hmargin=2.5cm,vmargin=1.5cm}
\usepackage{amssymb, amsmath,amsfonts,amsthm,mathrsfs}
\usepackage{hyperref}
\sloppy

\title{Rapport du projet de \emph{Web Data Management}}
\author{Shendan Jin \& Olivier Marty}
\date\today

\begin{document}

\maketitle

\section{Présentation du projet}

Le but de ce projet est d'utiliser différentes sources d'information sur
internet qui concernent l'état de différents moyens de transport (lignes de
métro, de train, stations vélo vide ou pleine, ou encore un bouchon sur
l'autoroute (non implémenté)) afin de prévenir l'utilisateur lorsque celui-ci va
les utiliser.
Pour cela on se connecte à son agenda (Google calendar), et on analyse ses mails
(Gmail) pour prévoir ses déplacements, mais d'autres sources pourrait être
ajoutées, de façon modulaire.

L'utilisateur est alors notifié par le medium de son choix (email, ou sms via
l'API de free mobile).

\section{Déscription technique}

Le projet est divisé en trois principales composantes :

\subsection{La classe event}

\subsection{La classe source} Cette classe représente une information provenant
d'internet et qui concerne l'état d'un moyen de transport.
Chaque instance doit fournir la méthode problem() (renvoit true si le moyen de
transport encontre un problème), et les attributs id (identifiant de la source)
et message (descriptif du problème).

Ces instances sont produites par plusieurs fonctions qui vont chercher
l'information idoine sur internet. Nous avons implémenté :

\paragraph{ratp\_trafic} En scrappant l'url \url{http://ratp.fr/meteo/}, cette
fonction est un générateur qui donne une source pour chaque ligne de la RATP.
(par exemple \{id="ligne\_rer\_B", message=""\}). % TODO message
Il est à noté que la ratp ne fournit aucune API publique.

\paragraph{transilien} Idem, en scrappant l'url \url{http://www.transilien.com/info-trafic/temps-reel},
cette fonction génère une liste de sources pour les lignes de la SNCF transilien
(par exemple \{id="RER-C", message=""\}). % TODO message
La SNCF fournit une API pour chercher des itinéraires, ou avoir les prochains
horaires, mais pas, à notre connaissance, pour avoir des informations de trafic.

\paragraph{jcdecaux\_vls} Cette fonction se connecte à l'API de jcdecaux\_vls
(les vélos libres services de jcdecaux, comme les Vélib', voir
\url{https://developer.jcdecaux.com/#/opendata/vls}).
Pour chaque station dont l'utilisateur est intéressé, elle génère deux sources :
une qui déclenche un problème lorsque la station est presque vide, et l'autre
lorsqu'elle est presque pleine (par exemple \{id="paris\_42707\_empty",
message="Station vélo 42707 - okabe (le kremlin-bicetre) à 18h53 le 20/02 : plus
que 2 places disponibles !"\} ou encore \{id="paris\_19001\_empty", message="Station
vélo 19001 - ourcq crimee à 18h58 le 20/02 : plus qu'un vélo disponible !"\})

\paragraph{} L'analyse de ces sources XML ou HTML est effectuée à l'aide de la
bibliothèque Python BeautifulSoup4.
Nous avions en premier lieu écrit des transformateurs en xquery pour formatter
toutes les données dans un schéma commun, mais l'utilisation d'un logiciel
externe (Saxon) rendait chaque requête très lente, d'autant que le résultat
n'était pas un objet natif Python.

\subsection{La boucle principale}

\end{document}
